\documentclass[onecolumn, 12pt]{IEEEtran}
\usepackage{amsfonts}
\usepackage{amsmath}
\usepackage{mathrsfs}
\usepackage{subfig}
\usepackage{algorithm}
\usepackage{algorithmic}
\usepackage{amssymb}
\linespread{1.2}\selectfont
\usepackage{amsthm}
\usepackage{hyperref}
\hypersetup{
  colorlinks=true,
  urlcolor=blue
}

% \newtheorem{thm}{Theorem}[section]
% \newtheorem{lem}{Lemma}[section]
% \newtheorem{prop}{Proposition}[section]
% \newtheorem{corollary}{Corollary}[section]
% \newtheorem{definition}{Definition}[section]

\newcommand\E[1]{\mathbb{E}\left[#1\right]}
\newcommand\EE[2]{\mathbb{E}_{#1}\left[#2\right]}
\newcommand\p[1]{\mathbb{P}\left(#1\right)}
\newcommand\Var[1]{\mathrm{Var}\left[#1\right]}
\newcommand\Cov[1]{\mathbb{Cov}\left[#1\right]}
\newcommand\Tr[1]{\mathrm{Tr}\left[#1\right]}
\newcommand\1[1]{\mathbb{I}_{\left\{#1\right\}}}
\newcommand\sign[1]{\mathrm{sign}(#1)}
\newcommand\cond[3]{#1_{#2|#3}}
\newcommand\N[2]{\mathcal{N}\left(#1, #2\right)}

\newtheorem{lemma}{Lemma}\newtheorem{theorem}{Theorem}\newtheorem{corollary}{Corollary}\newtheorem{remark}{Remark}\newtheorem{conjecture}{Conjecture}
\newtheorem{definition}{Definition}\newtheorem{proposition}{Proposition}
\newtheorem{example}{Example}

\begin{document}

\title{Reinforcement Learning}
\author{Rui Wu}
\maketitle

\section{Monte Carlo method and TD learning}

This section considers several methods for learning the optimal policy. We only consider episode environments.

Let $Q(s, a)$ be the state-action value function and $\pi(a|s)$ be the $\epsilon$-greedy policy derived from $Q(s, a)$. According to the policy improvement theorem, we start from an initial policy and keep improving it towards the optimal policy by iteratively estimating $Q(s, a)$ and updating $\pi(a|s)$. For each episode, we obtain the samples as
\begin{align*}
  s_0, a_0, r_1, s_1, a_1, r_2, s_2, \dots, r_T, s_T (, a_T).
\end{align*}
Then we estimate the sample value for each $(s_t, a_t)$ and update $Q(s, a)$ with these sample values.

\subsection{Batch Monte Carlo}
Monte Carlo method estimates $G_t$ as
\begin{align*}
  G_t = r_{t+1} + \gamma r_{t + 2} + \dots + \gamma^{T-t - 1}r_T. 
\end{align*}
For batch update, we first compute the average reward for each state-action pair as
\begin{align*}
  G(s, a) = \frac{\sum_t{G_t\delta_{s, a}(s_t, a_t)}}{\sum_t{\delta_{s, a}(s_t, a_t)}},
\end{align*}
and then update $Q(s, a)$ as
\begin{align*}
  Q(s, a) \leftarrow Q(s, a) + \alpha(G(s, a) - Q(s, a)). 
\end{align*}

\subsection{Online Monte Carlo}
Online Monte Carlo looks similar to TD learning and is easier to implement. It updates $Q(s, a)$ as
\begin{align*}
  Q(s_t, a_t) \leftarrow Q(s_t, a_t) + \alpha(G_t - Q(s_t, a_t)),
\end{align*}
where $G_t$ is defined as in batch Monte Carlo. Even though this method is called ``online'', it has to wait until an episode ends to actually compute the $G_t$'s.

\subsection{Sarsa}
It is similar to online Monte Carlo except that it computes $G_t$ as
\begin{align*}
  G_t = r_{t+1}+\gamma Q(s_{t+1}, a_{t+1}).
\end{align*}

\subsection{Q-learning}
It is similar to online Monte Carlo except that it computes $G_t$ as
\begin{align*}
  G_t = r_{t+1}+\gamma \max_a Q(s_{t+1}, a).
\end{align*}

\section{On policy and off policy learning}
TODO


\section{Eligibility trace}

This section focuses on appliying eligibility trace to $Q(s, a)$ than $V(s)$ and describes the $\text{Sarsa}(\lambda)$ algorithm in detail.

\subsection{Forward view of $\text{Sarsa}(\lambda)$}

We first generalizes the Sarsa reward $G_t$ to the $n$-step reward
\begin{align*}
  G_t^{(n)} = r_{t + 1} + \gamma r_{t + 2} + \dots + \gamma^{(n - 1)}r_{t +n} + \gamma^nQ(s_{t + n}, a_{t + n}), 
\end{align*}
and define
\begin{align*}
  G_t^\lambda = (1-\lambda)G_t^{(1)} + (1-\lambda)\lambda G_t^{(2)} + \dots + \lambda^{T - t} G_t^{(T - t)}. 
\end{align*}
If we assume an episode has infinite length and uss the assumption that the reward for a termination state remains the same, we can rewrite $G_t^\lambda$ as
\begin{align*}
  G_t^\lambda = (1 - \lambda) \sum_{n = 1}^\infty \lambda^{n - 1}G_t^{(n)}. 
\end{align*}
Then we update $Q(s, a)$ as
\begin{align*}
  Q(s, a) \leftarrow Q(s, a) + \alpha(G_t^\lambda - Q(s, a)). 
\end{align*}

This algorithm is between Monte Carlo and Sarsa. It incorporates a parameter $\lambda$ to balance how much into future we want to consider for reward estimation. However, to implement this forward view algorithm, we have to wait until an episode ends to start updating $Q(s, a)$, which is similar to online Monte Carlo.

\subsection{Backward view of $\text{Sarsa}(\lambda)$}
The backward view addresses the online update issue by introducing eligibility trace. The eligibility trace updates as
\begin{align*}
  Z(s, a) \leftarrow \lambda \gamma Z(s, a) + \delta_{s, a}(s_t, a_t),
\end{align*}
and the algorithm updates $Q(s, a)$ as
\begin{align*}
  Q(s, a) \leftarrow Q(s, a) + \alpha Z(s, a)(R_{t + 1} + \gamma Q(s_{t + 1}, a_{t + 1}) - Q(s_t, a_t)). 
\end{align*}

Note that this algorithm updates all states for each time $t$. This is very expensive compared to Sarsa or Monte Carlo. Is it really worth it?

\subsection{Connection between forward and backward views}
The forward and backward views are equivalent in an episode environments under first visit and batch update, i.e., $Q(s, a)$ is updated in batch after an episode ends. To illustrate this equivalency, we walk through the following example for policy evaluation.
\begin{example}
  Consider an episode as follows
  \begin{align*}
    s_1, a_1, r_2, s_2, a_2, r_3, s_1, a_3, r_4, s_4.
  \end{align*}
  Forward view. The estimated value for the first visit of state $s_1$ is
  \begin{align*}
    G^\lambda(s_1) =& (1 - \lambda) (r_2 + \gamma V(s_2)) \\
    & (1-\lambda) \lambda (r_2 + \gamma r_3  + \gamma^2 V(s_1))\\
    & \lambda^2(r_2 + \gamma r_3 + \gamma^2 r_4 + \gamma^3 V(s_4)).
  \end{align*}
  The value difference is
  \begin{align*}
    \Delta V_F(s_1) = G^\lambda(s_1) - V(s_1).
  \end{align*}
  Backward view. The eligibility trace for $s_1$ updates as
  \begin{align*}
    Z(s_1) = 1 \rightarrow \lambda\gamma \rightarrow 1 + \lambda^2\gamma^2.
  \end{align*}
  The value difference is
  \begin{align*}
    \Delta V_B(s_1) =& 1\cdot (r_2 + \gamma V(s_2) - V(s_1)) \\
    & \lambda\gamma \cdot (r_3 + \gamma V(s_1) - V(s_2))\\
    & (1 + \lambda^2\gamma^2)(r_3 + \gamma V(s_4) - V(s_1)).
  \end{align*}
  It is not difficult to show that $\Delta V_F(s_1) = \Delta V_B(s_1)$. However, this equality is not obvious from the first glance.
\end{example}

\subsection{Other eligibility trace algorithms}
There is no $\text{MC}(\lambda)$ as Monte Carlo method does not do bootstrap. It is possible to do $Q(\lambda)$ but it seems more involved. 


\section{Policy gradient, imitation learning and DAgger}

This section discusses the methods fro learning policies directly without first approximating the value function $Q(s, a)$. A policy parameterized by $\theta$ is denoted as $\pi_\theta(a|s)$. The idea is to create a reward function with the policy and find a good solution for $\theta$ by optimization techniques like stochastic gradient descent.

Note that as the algorithms do not involve $Q(s, a)$, the estimates are all based on the Monte Carlo methods rather than TD learning. In addition, eligibility trace does not apply in these algorithm. 

\subsection{Policy gradient}

Policy gradient uses the expected reward as the reward function,
\begin{align*}
  J(\theta) = \EE{s_0}{V^{\pi_\theta}(s_0)},
\end{align*}
where the expectation is taken over the initial state $s_0$. By expanding $V^{\pi_\theta}(s_0)$ over $a_0$, we get
\begin{align*}
  J(\theta) =& \EE{s_0}{\EE{\pi_{\theta}(a_0|s_0)}{Q^{\pi_\theta}(s_0, a_0)}}\\
  =& \EE{s_0}{\sum_{a_0}{\pi_{\theta}(a_0|s_0)Q^{\pi_\theta}(s_0, a_0)}}.
\end{align*}
The policy gradient theorem states that (not obvious at all as there are two components depending on $\theta$)
\begin{align*}
  \nabla_\theta J(\theta) =& \EE{s_0}{\sum_{a_0}{\nabla_\theta\pi_{\theta}(a_0|s_0)Q^{\pi_\theta}(s_0, a_0)}}\\
  =& \EE{s_0}{\EE{\pi_{\theta}(a_0|s_0)}{Q^{\pi_\theta}(s_0, a_0)\nabla_\theta\log\pi_\theta(a_0|s_0)}}.
\end{align*}
The gradient $\nabla_\theta J(\theta)$ is an expectation over the state-action pair and can be estimated by the Monte Carlo method. In particular, for an episode
\begin{align*}
  s_0, a_0, r_1, s_1, \dots, r_T, s_T(, a_T)
\end{align*}
the REINFORCE algorithm estimates $Q^{\pi_\theta}(s_0, a_0)$ with $G_t$ and updates $\theta$ as
\begin{align*}
  \theta \leftarrow \theta + \alpha G_t\nabla_\theta\log\pi_\theta(a_t|s_t).
\end{align*}

\subsection{Learning from experience}

Supervised learning has been used for many reinforcement learning problems and the primary difference is that reinforcement learning allows the agent to learn from its own experience. In this section we intentionally blur the line between supervised learning and reinforcement learning and consider a general formulation for learning from experience.

The experience is assumed to be provided of the form ${(s_t, a_t, G_t)}_{t = 1}^T$, where $G_t$ quantifies how good the state-action pair $(s_t, a_t)$ is. Let $f$ be the similarity function between two action distributions. Then we can define the reward function for a policy as
\begin{align*}
  J(\theta) = \sum_{t=1}^T{G_tf(\pi_\theta(a|s_t), \delta_{a_t}(a))}, 
\end{align*}
and the corresponding stochastic gradient update for $\theta$ is
\begin{align*}
  \theta \leftarrow \theta + \alpha G_t\nabla_\theta f(\pi_\theta(a|s_t), \delta_{a_t}(a)).
\end{align*}

\subsection{Imitation learning and DAgger}
Imitation learning is one example of learning from experience. The experience ${(s_t^*, a_t^*, 1)}_{t = 1}^T$ is completely generated by an expert policy and all demonstrations by the expert are considered equally good. The similarity function is ususally chose as the negative KL-divergent for discrete actions and negative mean squared error for continuous actions.

DAgger is similar to imitation learning except that the states in the experience ${(s_t, a_t^*, 1)}_{t = 1}^T$ are generated by the policy itself.

\subsection{Policy gradient as learning from experience}

In Andrej Kapathy's blog, policy gradient is described as learning from experience. Consider using the experience of an episode generated by the current policy. Let $G_t$ be the Monte Carlo estimated reward and the similarity function $f$ be the negative KL-divergence, i.e.,
\begin{align*}
  f(\pi_\theta(a|s_t), \delta_{a_t}(a)) = -D(\delta_{a_t}(a)||\pi_\theta(a|s_t)) = \log \pi_\theta(a_t|s_t),
\end{align*}
then the update for learning from experience reduces to policy gradient. However, given the vastly different derivations for policy gradient and learning from experience, this connection is more of a coincidence due to the functional form of the KL-divergence than some inherent relationship.


\section{A3C}
TODO

\end{document}

\section{Level}

What are the most urgent things to do? Huankai's diagram.

What are the most impactful things to do?

\section{What do we do}

We provide a general purpose optimization framework as a service. Our
customer is supposed to be user-facing and aims to optimze some metric with respect
to the traffic. For each project, the customer specifies a set of
actions to take and a target metric to optimize. In addition, the
customer provide the data collected from the users. Then our framework
learns a model from the historical data and makes real time decisions
on the actions to take.

\subsection{Discussion}

We assume the customer is a web service provider and the users visit
the service through browsers on various devices (PC, tablet, smart
phone and so on).

The data is a collection of user visit logs. In each visit, the user
information and the action taken are recorded. There are several types
of user information:
\begin{enumerate}
\item Basic information such as browser, device and ip address (including information that can be inferred from the ip address)
\item User account information when log in is required
\item Ads like information used in retargetting
\item User behavior after the action
\end{enumerate}
The realized metric or reward is computed from the user behavior after the
action. However, the metric is not necessarily associated with only
one visit. For example, it could happen that the user took several
visits and finally bought some product from the website. As another
example, the metric representing the number of visits of a user in a
week is defined over multiple visits.

The model learning extracts the repeating pattern from the data. The
first thought is to use visits as repeating units. However, as hinted
above, each visit is too rigid with respect to the metric of interest
thus might not be very informative. This fact opens a whole new
dimension of freedom on dividing the data and we propose the notion of
session as the repeating unit.

A session can be an arbitrary collection of visits. Given the
rationale of defining the session, we only consider metrics that are
based on the visits within a session for now. (If the metric depends
on multiple sessions, then we probably should redefine what a session
is.) In practice, we also require the session to be easily
deliminated. For simplicity, we assume only one metric is evaluated for a
session.

(However, with the definition of session, the notion of action becomes
unclear, as a session normally contains multiple actions. It seems we
should introduce the notion of super action that includes a series of
actions. When makeing a decision for the current action, we should
also look at the past actions we already took. What is fat
tuple? Is fat tuple trying to solve this issue? Is it solved?

To move on with the discussion, we assume each action is independent
of past actions in the same session. For simple experiments like the
SI web experiment, a session is simply a visit and the metric is
also defined within a visit. )

Now we can state the problem more formally. The data contains $N$
sessions. For each session, an action $a$ is taken and a realized
metric or reward $r$ is observed. In addition, we define a set of
features based on the session capturing the context $ctx$. The
definition of features can be quite flexible. Suppose there are
multiple visits each with the information of the browser, we can
either take empirical distribution of the browsers or just take the
last one. The processed data looks like ${(a_i, r_i,
  ctx_i)}_{i = 1}^N$. Then for a new context $ctx$, we need to decide
what action $a$ to choose in order to maximize the reward $r$.

(If past actions in the session matters, we also need to include some
features for the past actions and the formulation becomes much
more complicated. )

TODO: Our model does not depend on the goal, but by introducing this
dependency, one might be able to achieve better performance.










The data consists of




\subsection{Some thoughts}

We currently assume the data are about human users thus our customer
is likely to be a web service provider. Are there other scenarios our
framework can be applied?

Usually in live, we already have a notion of session, either created
by the customer or defined by us by default. This should not restrict
our exploration of the definition of a session. In fact, we can
redefine session, i.e., data split, and train a new model from
it. Note that the fact we used some definition of session to collect
the data does not prevent us from reusing the data. This might help us
identify better definition of session (whether a metric is optimizable
or not also depends on the choice of session).

In general, we assume projects are independent, i.e., when changing
the actions or the target metric requires retraining the
model. However, since we model all these variables with a joint
distribution, so we have some capacity of reusing the existing model
when the customer makes changes on the project.



In each
use this service, the customer creates a project, in which the
customer specifies a metric

Online training -> offline sanity check as in Bob fishing.

\section{Distributions}

\pagebreak

\subsection{Discrete}

\href{http://www.keithschwarz.com/darts-dice-coins/}{Alias method} for
efficient sampling.

It is still problematic as the distribution keeps changing. See
\href{http://stackoverflow.com/questions/25189406/c-discrete-distribution-sampling-with-frequently-changing-probabilities}{here}. The
paper \href{http://theory.stanford.edu/~matias/papers/MVN03.dynamic_rv_gen.pdf}{Dynamic Generation of Discrete Random Variates}
is some literature.

\subsection{Continuous}

There are two types of parameterization for the normal-gamma
distribution. According to \href{https://en.wikipedia.org/wiki/Normal-gamma_distribution}{Wikipedia} and the \href{https://www.cs.ubc.ca/~murphyk/Papers/bayesGauss.pdf}{note}, the first type of
parameterization $NG(\mu, \lambda, \alpha, \beta)$ performs the
posterior update as follows:
\begin{align*}
  \mu = & \frac{\lambda_0\mu_0 + \sum_ix_i}{\lambda_0 + n}\\
  \lambda = & \lambda_0 + n\\
  \alpha = & \alpha_0 + \frac{n}{2}\\
  \beta = & \beta_0 + \frac{1}{2}\sum_i(x_i - \bar{x})^2 + \frac{\lambda_0 n (\bar{x} -
            \mu_0)^2}{2(\lambda_0 + n)}.
\end{align*}
The second type of parameterization $NG(w_s, w_l, m_1, m_2)$ is much
simpler to implement in practice and it performs
the posterior update as follows:
\begin{align*}
  w_s = & w_{s0} + n\\
  w_l = & w_{l0} + n\\
  m = & m_0 + \sum x_i\\
  s = & s_0 + \sum x_i^2.
\end{align*}
\begin{theorem}
If $(\mu_0, \lambda_0, \alpha_0, \beta_0)$ and $(w_{s0}, w_{l0},
m_0, s_0)$ satisfy the following consistency condition
\begin{align*}
  w_s = \lambda,\ w_l = 2\alpha,\ m = \lambda\mu,\ s
  = 2\beta + \lambda\mu^2,
\end{align*}
or equivalently,
\begin{align*}
  \mu = \frac{m}{w_s},\ \lambda = w_s,\ \alpha =
  \frac{w_l}{2},\ \beta = \frac{s}{2} - \frac{m^2}{2w_s},
\end{align*}
then the above two posterior updates are equivalent and the
parameters always satisfy the consistency condition.

\end{theorem}

\begin{proof}
Below we assume the first three conditions are satisfied and only
verify the consistency of the $\beta$ update.
\begin{align*}
  \beta = & \beta_0 + \frac{1}{2}\sum x_i^2 - \frac{1}{2}n\bar{x}^2 +
            \frac{\lambda_0 n}{2(\lambda_0 + n)}\left[\bar{x}^2 -
            2\bar{x}\mu_0 + \mu_0^2\right]\\
  = & \beta_0 + \frac{1}{2}(s - s_0) + \frac{1}{2(\lambda_0 + n)}\left[-n^2\bar{x}^2 -
            2\lambda_0\mu_0n\bar{x} + \lambda_0n\mu_0^2\right]\\
  = & \beta_0 + \frac{1}{2}(s - s_0) - \frac{1}{2w_s}(n\bar{x} +
      \lambda_0\mu_0)^2 + \frac{1}{2(\lambda_0 +
      n)}(\lambda_0^2\mu_0^2 + \lambda_0n\mu_0^2)\\
  = & \frac{s}{2} - \frac{m_s^2}{2w_s} + \beta_0 - \frac{s_0}{2} +
      \lambda_0\mu_0^2\\
  = & \frac{s}{2} - \frac{m_s^2}{2w_s}.
\end{align*}
\end{proof}

When we have $w_{s0} = \epsilon, w_{l0} = \epsilon, m_0 = 0, s_0 = 1$, we get $\mu_0
= 0, \lambda_0 = \epsilon, \alpha_0 = \epsilon, \beta_0 = 1/2$. As $\lambda \approx 0$ he Gamma distribution
concentrate around $0$, the normal distribution for the mean behaves
like $N(0, \infty)$. The predictive distribution is
$t_{2\epsilon}(x|0, \frac{1+\epsilon}{2\epsilon^2})$ where the second
parameter is the variance not the precision (see (110) in
\href{https://www.cs.ubc.ca/~murphyk/Papers/bayesGauss.pdf}{note}). It
roughly means $\epsilon x$ is a t-distribution with degree of freedom
$2\epsilon$.

\pagebreak


Here is the command to plot it in R.

graphics.off(); x <- seq(-4, 4, length = 100); epsilon <- 0.001; plot(x / epsilon, dt(x, epsilon))




\subsection{Fixed bins}

The posterior update is suspicious. The outside samples are ignored
somehow. Is this consistent?

\subsection{Bin}


Bin is based on an implementation of
\href{https://github.com/tdunning/t-digest/blob/master/docs/t-digest-paper/histo.pdf}{TDigest}. The
algorithm (Algorithm 1) is easy to follow. Not sure why this paper is not
properly published and has no citation.

Algorithm 2 is about computing the CDF. The way it defines the uniform
distribution is very suspicious.

Bins are interpretated as mixture of uniform distributions. The
intervals of the uniform distributions are defined by
HalfIntervals. Does the CDF respect this definition?

The divergence between two bins are defined by ``align and
compute''. Bins reserves some unseen weight as $\alpha$

Quantized is not used much currently. Should look for bins instead.




\subsection{SparseTuple}

Tuple is the product of distributions. When sampling from a tuple,
each dimension is sampled. SparseTuple in addition has a total weight $w$ and a
weight for each dimension $w_d$. When sampling from a sparseTuple,
dimension $d$ is sampled with probability $w_d / w$.

\section{Distance}

\subsection{Levy-Prokhorov metric}

This metric is used to measure distance between distributions in
probability theory. However is does not look very computable. The
computation in one dimension is discussed
\href{http://math.stackexchange.com/questions/224951/algorithms-for-computing-or-numerically-approximating-the-prokhorov-metric}{here}. On
the other hand, for one dimensional continuous distributions,
Kolmogorov�CSmirnov test seems to be a similar metric. They seem to be equally
hard to compute. For one dimensional Gaussians, there might be close
form solution. For mixture distributions or mixture with PGM, sampling might be the only
way to compute.

It seems we can construct examples where importance sampling does not
help.

It is mentioned \href{https://en.wikipedia.org/wiki/Kolmogorov%E2%80%93Smirnov_test}{here} that ``In practice, the statistic requires a relatively large number of data points to properly reject the null hypothesis.''

\subsection{KL divergence for mixture models}

MixtureDivergenceVariationalBound ``APPROXIMATING THE KULLBACK LEIBLER DIVERGENCE
BETWEEN GAUSSIAN MIXTURE MODELS''.



Verify equation (20) in the paper. The optimal value of $L_f(f,
\varphi)$ has a simple experession.

What is used as the similarity between component $a$ and $b$? Why is
it better than the Jensen-Shannon divergence?

It somehow makes sense. Proportional to the component weight and
$\exp(-D(f_a||g_b))$.

In experiment, when the mixture model shows $D(f_a||f_b) \geq 0.5$,
noticing that $D(f_a||f_a) = 0$, the iteration is able to drive the
$\phi$ matrix as a diagonal one.
\begin{remark}
  Here is a rough argument of the above observation. Consider the
  ideal case where $D(f_a||f_b)$ are small and the
  mixture model has equal weights on all components. Start with
  $\varphi$ an all-one matrix. First $\phi$ is
  $\exp(D)$ with row norm. As the $i$-th row of $\exp(D)$ has the $i$-th term equal to
  $1$ and all other terms are close to $0$. Thus the normalization
  does not matter that much and $\phi ~ \exp(D)$. On the other hand $\varphi$
  is $\phi$ with column normalization and we can argue similarly that
  $\varphi ~ \exp(D)$. Then in the $n$-th iteration, $\phi ~ \exp(nD)$
  and it will converge to the diagonal matrix.

  In the case when the mixture weights are not exactly equal and the
  off-diagonal terms of $D$ are not that small, the exponential
  decaying iteration should only be affected up to some constant
  factor and we expect the result still being diagonal.
\end{remark}

For mixture of Gaussian, the Jensen-Renyi divergence with $alpha = 2$ has a closed form
solution.
\href{https://www.cise.ufl.edu/~anand/pdf/jr-miccai09-final.pdf}{Closed-form Jensen-Renyi Divergence for
Mixture of Gaussians \& Applications to
Group-wise Shape Registration}.
\begin{enumerate}
\item Is this generalizable to other
distributions? It seems (4) is trackable when Gaussian is replaced
with a Tuple.
\item Do we really need a divergence for mixture
distributions? For the latter question, we could use this divergence
to quantify the distance between the oracle mixture model and the
learned mixture model.
\end{enumerate}

Oh, there is another closed-form expression for some divergence of
Gaussian mixture model in ``Closed-Form Expression for Divergence of Gaussian Mixture Models''.





\section{Posterior Update}

\section{EM}
There are several local optimums that we are thinking of. First, a
component is dead. Second, two components that are the same and both
captures some component. Third, two components that are similar and
jointly captures some component.

\section{Information bottleneck}

The original paper for EM information bottleneck is
\href{http://www.cse.huji.ac.il/~nir/Papers/Elidan05.pdf}{here}.

There are two things I want to know. What are the followup works for
it? How are people doing evaluation?

ITM is partly derived from this. How does it fit into the big picture?


\subsection{Rate distortion theory}
The goal of rate distortion theory is understand how to compress a
source $X$ with a desired distortion. As in the channel coding case,
it is more efficient to consider the compression in the block coding
sense for $n$ i.i.d. symbols $X^n$. The distortion function $d$ on
multiple symbols is defined as
\begin{align*}
  d(X^n, \hat{X}^n) = \sum_i d(X_i, \hat{X}_i).
\end{align*}
The rate distortion problem is
defined as, for a desired distortion level $D$, find the minimum rate
$R$, such that there exists an encoding
function
\begin{align*}
  f_n : X^n \to \{1, \dots, 2^{nR}\}
\end{align*}
and an decoding function
\begin{align*}
  g_n : \{1, \dots, 2^{nR}\}\to \hat{X}^n
\end{align*}
satisfying $Ed(X^n, g_n(f_n(X^n)))\leq D$. Thus we can define the rate
distortion function as
\begin{align*}
\bar{R}(D) = \min_{p(\hat{x}^n|x^n):Ed(X^n, \hat{X}^n)\leq nD} I(X^n;
  \hat{X}^n).
\end{align*}
The achievability of this definition is understood as
follows. Consider a compression mapping $p(\hat{x}^n|x^n)$ which
satisfies the distortion constraint. There are $2^{H(X^n)}$ typical
sequences for $X^n$. The encoding function $f_n$ maps an
$x^n$ to an $\hat{x}^n$ that is jointly typical with $x^n$ and the
decoding function $g_n$ is simply identity. There are $2^{H(X^n,
  \hat{X}^n)}$ jointly typical sequences and $2^{H(\hat{X}^n)}$
typical sequences for $\hat{X^n}$, thus each $\hat{x^n}$ is
jointly typical with $2^{H(X^n|\hat{X}^n)}$ typical sequences for
$X^n$, therefore at least $2^{I(X^n;\hat{X}^n)}$ typical sequences for
$\hat{X^n}$ is required to represent all the typical sequences for
$X^n$.

A remarkable result is to show that the above definition is equivalent
to the single symbol definition
\begin{align*}
R(D) = \min_{p(\hat{x}|x):Ed(X, \hat{X})\leq D} I(X;
  \hat{X}).
\end{align*}
\begin{theorem}
  $\bar{R}(D) = nR(D)$.
\end{theorem}
\begin{proof}
  By definition, $\bar{R}(D) \leq nR(D)$. Next we show that $\bar{R}(D) \geq nR(D)$. Following the proof of \cite[Theorem 10.2.1]{ElementsOfIT},
  \begin{align*}
    \bar{R}(D) \geq &\min_{\prod_ip(\hat{x}_i|x_i):Ed(X^n, \hat{X}^n)\leq nD} \sum_iI(X_i;
  \hat{X}_i)\\
    \geq & \min_{D^n: \sum_i D_i = nD}\sum_iR(D_i).
  \end{align*}
  As show in \cite[Lemma 10.4.1]{ElementsOfIT}, $R(D)$ is a convex
  function thus
  \begin{align*}
    \sum_iR(D_i) \geq nR(\frac{1}{n}\sum_iD_i).
  \end{align*}
  Therefore $\bar{R}(D) \geq nR(D)$.
\end{proof}

\subsection{Motivation}

To motivate the definition of information bottleneck, we rewrite the
optimization of the rate distortion optimization as follows
\begin{align*}
\min_{p(\hat{x}|x)} I(X;
  \hat{X}) + \lambda Ed(X, \hat{X}).
\end{align*}
Instead of minimizing the distortion, the information bottleneck
approach aims to maximize the information $\hat{X}$ conveys about some
other variable $Y$, therefore \cite{Tishby99} introduces
the following optimization problem
\begin{align*}
\min_{p(\hat{x}|x)} I(X;
  \hat{X}) - \lambda I(Y; \hat{X}).
\end{align*}


Multivariate information bottleneck

Continuation and deterministic annealing -- this is quite relevant to
our split trainer.

\subsection{$P$ and $Q$ distribution}

Let $X$ be the observed variables. We model $X$ using a family of mixture distributions $P_\theta$ with $C$ being the hidden component label. The goal is estimate the optimal $\theta$ from the samples $\{X^{(n)}\}_{n = 1}^N$,
$$\theta^* = \arg\max_\theta \sum_{n = 1}^N \log P_\theta(X^{(n)}). $$

Let $Q(x)$ be the empirical distribution of the samples, i.e.,
$$ Q(X = x) = \frac{1}{N}\sum_{n = 1}^N \delta_{X^{(i)}}(x),$$
where $\delta_x(\cdot)$ is the delta function for either a discrete or continuous alphabet. Let $P_\theta(X, C)$ be the model distribution. Then the EM algorithm estimates $\theta^*$ by solving the following optimization problem
$$\max_{Q(c|x), \theta}\sum_xQ(x)\sum_cQ(c|x)\log\frac{P_\theta(x, c)}{Q(c|x)}, $$
and it converges to a local optima that satisfies the fixed point conditions:
$$ Q(c|x) = P_\theta(c|x), \quad \sum_{x, c}\frac{Q(x)}{P_\theta(x)}\frac{\partial}{\partial\theta}P_\theta(x, c) = 0. $$
When $P_\theta(x)$ is allowed to be any distribution, then the second condition becomes $P_\theta(x) = Q(x)$. In general, when some variable is continuous, as $Q(x)$ is a spiky distribution and $P_\theta(x)$ is usually assumed to be continuous, they are not always equal. However, we expect the second condition to imply $P_\theta(x)$ is close to $Q(x)$ in the \href{https://en.wikipedia.org/wiki/L%C3%A9vy%E2%80%93Prokhorov_metric}{Prokhorov metric}. Therefore we can roughly use $P$ and $Q$ interchangeably in our derivation.


\subsection{How to solve it}

\subsection{Density exponent}

IB with $\beta$ optimizes
\begin{align*}
  \max_Q -I(X;C) + \sum_i \beta_i I(X_i;C),
\end{align*}
which is equivalent to
\begin{align*}
  \max_Q H(X|C) - \sum_i\beta_iH(X_i|C).
\end{align*}
Below we discuss the effect of $\beta$. The claim is that, for
relevant context variables, $\beta$ must be nonzero in order to learn anything
useful.

Consider an example with two variables $(Y, R)$ which are dependent. If we set $\beta_R = 1$ and $\beta_Y =
0$, IB becomes
\begin{align*}
\max_Q H(Y|R, C).
\end{align*}
It is easy to see that $C = f(R) \in \mathbb{N}$ for any deterministic function $f$
is a global optima to the above optimization problem, thus a global
optima to IB. The EM algorithm could randomly terminate on any such
global optima. In particular, $C \equiv 1$ or single component is a
global optima.

\subsection{Entropy estimation}



\href{https://arxiv.org/pdf/cond-mat/0305641.pdf}{This} proposed a
method for estimating the mutual information of continuous
variables. It is based previous works for entropy estimation
[19]. Note that equation (4) is another method for entropy estimation
using spacing. However, it only works for one dimensional distribution
and does not generalize to higher dimensions.

The basic idea in [19] as reviewed in Section II.A is as
follows. Assume $X\in \mathbb{R}^D$ and there are $N$ observations. As
\begin{align*}
  H(X) = -\frac{1}{N}\sum_i\log\mu(x_i),
\end{align*}
it is sufficient to come up with an estimator of $\log\mu(x_i)$ for
any $i$. For a given distance function $||\cdot||$, let $d_{i, k}$ be the distance between $x_i$ and its $k$-th
nearest neighbor. Then we approximate $\mu(x_i)$ by assuming
$\mu(x)$ is constant within $B(x_i, d_{i, k}) = \{||x - x_i||<d_{i, k}\}$ and get
\begin{align*}
  \mu(x_i)\approx & \frac{\int_{B(x_i, d_{i,
                    k})}\mu(x)dx}{\int_{B(x_i, d_{i, k})}dx}\\
  = &\frac{\int_{B(x_i, d_{i,
                    k})}\mu(x)dx}{C_D(2d_{i, k})^D}.
\end{align*}
By definition of $\mu(x)$, the numerator is approximately $k/N$. Thus
\begin{align*}
  \log \mu(x_i)\approx \ln(k) - \ln(N) - \ln(C_D) - D\ln(2d_{i, k}).
\end{align*}
A better approximation for the numerator is considered in the paper,
which could be more precise on the boundary of $B(x_i, d_{i, k})$. It shows that
\begin{align*}
  \log \mu(x_i)\approx \psi(k) - \psi(N) - \ln(C_D) - D\ln(2d_{i, k}),
\end{align*}
where $\psi$ is the digamma function. We note that $\psi(z)$ and
$\ln(z)$ are very close when $z$ is a positive real number. In fact,
as $x \to \infty$, $\psi(x)$ gets arbitrarily close to both
$\ln(x-1/2)$ and $\ln(x)$
\href{https://en.wikipedia.org/wiki/Digamma_function}{here}.

Now we have the tool to understand why binning performs poorly. The
key question is about how we estimate $$ I_i =
\ln\frac{p(y_i|x_i)}{p(y_i)}. $$ To avoid bias induced by different
resolutions in the numerator and the denominator, the general method
is to pick a common neighborhood $B_i$ and estimate $I_i$ within
$B_i$. Clearly $B_i$ cannot be too big as large window size blurs the
joint distribution and seems to push $\hat{I}_i$ towards $0$. On the
other hand, $B_i$ cannot be too small either
and should contain sufficient number of samples to accurately estimate
$I_i$. In particular, if
$B_i$ only contains $y_i$ itself, we get $\hat{I}_i =
\ln\frac{N}{N_i}$ which losses all local information about the density
and causes a bias. The KNN method picks $B_i = B(y_i, d_{i, k})$ which
strikes a very nice balance. While by definition $B_i$ is contrained
to a local neighborhood, it also ensures that $p(y_i|x_i)$ has $k$
samples and $p(y_i)$ has at least $k$ samples. We should note that, as
KNN suffers from curse of dimension, we do not expect this method to
work well for high dimensional distributions. There are two types of
binning: fixed bins or adaptive bins. Fixed binning method
picks $B_i$ based on some grid. Without using the local
information, these bins are likely to be either too big or too small
for each sample, thus the overall estimation is unreliable. Adaptive
binning picks $B_i$ that equally splits the density. The number of
samples within a bin are usually the same and larger than one, so the
method will be fine, though there is some bias in approximating the
density (see \href{http://c4dm.eecs.qmul.ac.uk/papers/2009/StowellPlumbley09entropy.pdf}{here}).

\begin{remark}
  It is not hard to implement the algorithm in
  \href{http://journals.plos.org/plosone/article?id=10.1371/journal.pone.0087357#s5}{here}. However,
  it is easy to get the base of the logarithm wrong. The digamma
  function implicitly uses the natural logarithm which should also be
  used in numerical evaluation of the mutual information. I have
  noticed that there seems to be a factor bias of $2/3$ (not a random bias). Potentially I could
  figure it out in either of the following two ways, but I guess I have already made
  the right decision to move on.
  \begin{enumerate}
    \item Read the Matlab implementation. Luckily this paper has
      attached its implementation.
    \item Identify that the factor bias is in fact a single number. Reverse engineer the factor bias. It is still hard to
      realize the factor is $\ln(2)$ though.
    \end{enumerate}

\end{remark}

\href{http://jimbeck.caltech.edu/summerlectures/references/Entropy%20estimation.pdf}{This}
is a classical paper about entropy estimation.


\href{https://en.wikipedia.org/wiki/Differential_entropy}{Here} is
some discussion about differential entropy. Differential entropy is in
general not invariant under arbitrary invertible maps. The continuous mutual
information is usually more desirable than the differential entropy. A
modification of differential entropy that addresses these drawbacks is
the relative information entropy

\href{https://en.wikipedia.org/wiki/Limiting_density_of_discrete_points}{Here}
and
\href{https://books.google.com/books?id=tTN4HuUNXjgC&pg=PA375&lpg=PA375&dq=limiting+density+of+discrete+points&source=bl&ots=H5PusxHAS6&sig=ItDnzEQod_UffAWcvbPCC2eteTQ&hl=en&sa=X&ved=0ahUKEwjCoLbzhabPAhVP42MKHbX0DwIQ6AEIOzAF#v=onepage&q=limiting%20density%20of%20discrete%20points&f=false}{here}
is an alternative definition for continuous variables and a not so
useful discussion
about can be found
\href{http://lo-tho.blogspot.com/2013/01/the-wikipedia-article-for.html}{here}.

\href{http://papers.nips.cc/paper/1408-new-approximations-of-differential-entropy-for-independent-component-analysis-and-projection-pursuit.pdf}{This}
is a highly cited paper about differential entropy estimation. It
talks about how to approximate the maximum entropy density with finite
samples. Not very useful here.

\pagebreak

\section{Modeling missing values}

This section discusses how we handle missing values. We first
rigorously defines the probability triple and then show how we learn a
model from the samples. In this section, we use the term model and
probability interchangeably.

\subsection{Sample space}
Consider a random vector $X = (X_1, \dots, X_D)$ where $X_i\in
\mathcal{X}_i$. When $X_i$ is missing, it is denoted as $X_i =
\epsilon_i$. Then the space for random vectors with missing values is
defined as
\begin{align*}
  \bar{\Omega} = \prod_i\bar{\mathcal{X}}_i, \ \text{where}\
  \bar{\mathcal{X}}_i = \mathcal{X}_i\cup \{\epsilon_i\}.
\end{align*}
We call $\bar{\Omega}$ the instance space since each $\bar{x}\in
\bar{\Omega}$ is an instance in our implementation.

We can describe missing values from another perspective. For each $i$,
let $\phi_i\in\{0, 1\}$ be the random variable indicating whether
$X_i$ is missing, i.e., $\phi_i = 1$ if $X_i$ is missing. Introducing the new
variable $\phi_i$ decouples the value of $X_i$ and whether it is
missing, and gives us more flexibility in specifying the distribution
of a model. We thus define the model space as
\begin{align*}
  \Omega = \prod_i (\mathcal{X}_i\times \{0, 1\})
\end{align*}
and each element in $\Omega$ is denoted by $(x, \phi)$.

The model space $\Omega$ is larger than the instance space
$\bar{\Omega}$. In particular, we can define the equivalence
relationship $\sim$ such that two elements $(x^1, \phi^1) \sim (x^2,
\phi^2)$ if and only if $\phi^1 = \phi^2$ and $x_i^1 = x_i^2$ for any
$i$ such that $\phi_i^1 = 0$. It is easy to see that there is a
bijection between the quotient space $\Omega / \sim$ and
$\bar{\Omega}$. We define the quotient map $\tau: \Omega \to
\bar{\Omega}$ as
\begin{align*}
  \tau((x, \phi)) = \bar{x}, \ \text{where}\ \bar{x}|_{\phi = 0} =
  x|_{\phi = 0},\ \bar{x}|_{\phi = 1} = \epsilon|_{\phi = 1}.
\end{align*}

\subsection{Events}
We define the set of events to be the greatest $\sigma$-algebra on
both spaces, i.e., $\bar{\mathcal{F}} = 2^{\bar{\Omega}}$ and
$\mathcal{F} = 2^\Omega$.
There is a family of events on $\Omega$ that is of particular
interest. It is the set of instances,
\begin{align*}
  \mathcal{A} = \{A_{\bar{x}}\}_{\bar{x}\in\bar{\Omega}}, \
  \text{where}\ A_{\bar{x}} = \tau^{-1}(\bar{x}).
\end{align*}

\subsection{Probability}
Let $P$ denote a probability on $(\Omega, \mathcal{F})$. The
induced probability $\bar{P}$ of $P$ on $(\bar{\Omega}, \bar{\mathcal{F}})$
is defined as
\begin{align*}
  \bar{P}(\bar{x}) = P(\tau^{-1}(\bar{x})), \ \forall \bar{x}\in\bar{\Omega}.
\end{align*}
In fact, all probabilities on $(\bar{\Omega}, \bar{\mathcal{F}})$ can
be defined this way.

We consider two families of probabilities on $(\Omega,
\mathcal{F})$. The first family assumes $X$ and $\Phi$ are
independent,
\begin{align*}
  P(x, \phi) = P(x)P(\phi),
\end{align*}
thus $P(x)$ and $P(\phi)$ can be specified separately. For example,
$P(x)$ could be defined by some probabilistic graphical model while
$P(\phi) = \prod_iP(\phi_i)$ is a set of independent Bernoulli random
variables.

The second family assumes $X$ and $\Phi$ are independent given some
hidden variable $C\in [K]$. In this case, $P$ is defined on the
enlarged product space $\Omega \times [K]$,
\begin{align*}
  P(x, \phi, c) = P(c)P(x|c)P(\phi|c)
\end{align*}
and $P(x, \phi)$ is obtained by marginalizing over $C$. The sparse
tuple mixture model belongs to this family and has the form
\begin{align*}
  P(x, \phi, c) = P(c)\prod_iP(x_i|c)P(\phi_i|c).
\end{align*}

In general, $\bar{P}$ induced by $P$ only identifies the probability
$(\Omega, \mathcal{A}, P)$. However, when there is more structure on
$P$, it might be possible to uniquely identify $P$ from $\bar{P}$. The
sparse tuple mixture model is one such example. For a give $P$,
\begin{align*}
  \bar{P}(\bar{x}, c) = P(c)\prod_i\bar{P}(\bar{x_i}|c)
\end{align*}
where $\bar{P}(x_i|c) = P(x_i|c)P(\phi_i = 0|c)$ for any $x_i\in
\mathcal{X}_i$ and $\bar{P}(\epsilon_i|c) = P(\phi_i = 1|c)$. On the
other hand, for a given $\bar{P}$,
\begin{align*}
  P(x, \phi, c) = \bar{P}(c)\prod_iP(x_i|c)P(\phi|c)
\end{align*}
where $P(x_i|c) = \frac{\bar{P}(x_i|c)}{1 - \bar{P}(\epsilon_i|c)}$
for any $x_i\in\mathcal{X}_i$ and $P(\phi_i = 1|c) =
\bar{P}(\epsilon_i)$.

\subsection{Learning a sparse tuple mixture model from instances}

The learning task is as follows: given a set of instances generated
from a true model $(\Omega, \mathcal{F}, P^*)$, find a model in a
given family of models that best approximates $P^*$.

We first note that, without any restriction on the true model $P^*$, even
when there is no constraint on the family of models we can use, it is
not possible to recover $P^*$ with infinite number of
instances. The reason is that the instances belong to the space
$\bar{\Omega}$ and the part of $P^*$ that is not captured by
$\bar{P}^*$ is not observable from the instances. Therefore we can
only hope to approximate $(\Omega, \mathcal{F}, P^*)$ up to
$(\bar{\Omega}, \bar{\mathcal{F}}, \bar{P}^*)$ or $(\Omega,
\mathcal{A}, P^*)$. If there is a one to one map between $P^*$ and
$\bar{P}^*$, such as when $P^*$ is a sparse tuple mixture model, then
$P^*$ can be approximated from instances. However, we should not make
this assumption in real applications where $P^*$ is unknown.

The current family of models we use for learning is the sparse tuple
mixture model. It is not difficult to write down the modified
stochastic variational inference algorithm for such models (it is
modified because the E-step is following the EM algorthm than the
stochastic variational inference algorithm). In our current
implementation, the E-step handles the missing values in a wrong way
but the M-step is correct.

Assume we use the sparse tuple mixture model for learning. For
simplicity we represent the model in $\bar{\Omega}$ with a
hyperparameter $\theta$. The goal is to estimate the optimal $\theta$
from the instances $\{\bar{x}^{(n)}\}_{n = 1}^N$,
\begin{align*}
  \theta^* = \arg\max_\theta\sum_{n =
  1}^N\log\bar{P}_\theta(\bar{x}^{(n)}).
\end{align*}
When it is clear from the context, we will omit $\theta$ in the
notation and simply write $\bar{P}_\theta$ as $\bar{P}$. Let
$Q(\bar{x})$ be the empirical probability of the instances, i.e.,
\begin{align*}
  Q(\bar{x}) = \frac{1}{N}\sum_n\delta_{\bar{x}^{(n)}}(\bar{x}),
\end{align*}
where $\delta$ is the dirac distribution for either discrete or
continuous alphabet. Note that $Q\to \bar{P}^*$ when $N\to
\infty$. The EM algorthm estimates $\theta^*$ by solving the following
optimization problem with alternating maximization,
\begin{align*}
  \max_{Q(c|\bar{x}),
  \theta}\sum_{\bar{x}}Q(\bar{x})\sum_cQ(c|\bar{x})\log\frac{\bar{P}_\theta(\bar{x},
  c)}{Q(c|\bar{x})}.
\end{align*}
The algorithm converges to a local optima that satisfies the fixed
point condition
\begin{align*}
  Q(c|\bar{x}) = \bar{P}_\theta(c|\bar{x}), \ \sum_{\bar{x}, c}\frac{Q(\bar{x})}{\bar{P}_\theta(\bar{x})}\frac{\partial}{\partial\theta}\bar{P}_\theta(\bar{x}, c) = 0.
\end{align*}

When $\bar{P}(\bar{x})$ is allowed to be any distribution, the second
condition becomes $\bar{P}(\bar{x}) = Q(\bar{x})$. In general, when
some variable is continuous, as $Q(\bar{x})$ is a spiky distribution
while $\bar{P}_\theta(\bar{x})$ is usually from some smooth family,
they are not always equal. If $P^*$ is close to the family of sparse
tuple mixture model, we conjecture that the second condition implies
that $\bar{P}_\theta(\bar{x})$ is close to $Q(\bar{x})$ in the
\href{https://en.wikipedia.org/wiki/L%C3%A9vy%E2%80%93Prokhorov_metric}{Prokhorov metric}.
One easy mistake is to claim that $P^*$ is always close to some sparse
tuple mixture model as this family is dense in the space of
probabilities. There are two problems with this claim. Firstly, this family is
only dense in $\bar{\Omega}$, not in $\Omega$. Secondly, in current
practice, we only consider sparse tuple mixture model with only finite
number of components, which is not a dense set even in
$\bar{\Omega}$. Therefore, we do not always expect our model to be
able to capture the true model $P^*$.

\subsection{Inference related quantities}

The optimization objective is always easy to estimate. It is simply
$$
\sum_{\bar{x}}Q(\bar{x})\log\bar{P}(\bar{x}),
$$
which does not rely on the fixed point condition above and can be
evaluated for any $\bar{P}$. It is a useful metric to monitor the
progress of the optimization. When the true model $P^*$ is known, we
can plug it in and get an upperbound of the objective. However, when
the true model $P^*$ is not known, it would be interesting to know
how the upperbound could be approximated.

For single dimension, it is not hard to see that
\begin{align*}
  H(\bar{X}_i) = H(\Phi_i) + P(\Phi_i = 0)H(X_i).
\end{align*}

Below we briefly discuss the estimation of some high dimensional
quantities. It is still unclear how useful these quantities are.
\begin{enumerate}
\item $H_{P^*}(\bar{X}_S)$ could be estimated using all samples.
\item $H_{P^*}(X_S|\phi_S = 0)$ could be estimated using the samples where
  no data is missing.
\item $H_{P^*}(X_S)$ is not observable in general. It is only
  observable for models where $\bar{P}^*$ can uniquely identify $P^*$,
  such as the model in which $X$ and $\Phi$ are independent or
  independent given some latent variable (e.g., sparse tuple mixture
  model). When $P^*$ is a sparse
  tuple mixture model, we can first learn a model $P$ from the instances
  and then estimate $H_{P^*}(X_S)$ by $E_Q[-\log P(X_S)]$. The estimation is
  good when $P$ is close to $P^*$.
\item $H_P(\bar{X}_S)$, $H_P(X_S|\Phi_S = 0)$ and $H_P(X_S)$ are
  possible to estimate given $P$, though the mixture model is usually not
  tractable and there is issue about high
  dimensional computation.
\item $E_{P^*}[f(P(X))]$ can be approximated by $E_Q[f(P(X))]$. The optimization
  objective is one such example.
\end{enumerate}

\pagebreak


\section{Contextual multi-arm bandit}

\href{https://arxiv.org/pdf/1508.03326v2.pdf}{This} is a useful survey
about this topic. Section 4.1.4 is relevant to what we do.

The main question is, does soft split make sense? As the policy
$p(a|x)$ is not determined by the environment, the hard split conditional distribution
$p(x, r|a) = p(x|a)p(r|x, a)$ is not fully determined by the
environment. How do we handle this?

One possibility is to build two
layers of mixtures, first on $x$, then on $a$, thus the first layer is
about $p(x)$ and the second layer is about $p(r|x, a)$. Is this the
right way to do it? One question to ask here is that, how should we
cluster $x$? We could cluster $x$ by itself, but we could also cluster
$x$ with respect to $a$ and $r$. How to do this in a more principled way?


\section{Lift tree}
Lift tree is closely related to the decision tree, which is a diverge
from our core algorithm based on mixture model. Both
algorithms construct a tree by first picking a set of variables
according to some optimization criterion and then
generating new tree nodes for different configurations of these
variables. Here is the algorithm for lift tree.
\begin{enumerate}
\item Specify the candidate sets of variables. In our case, due to
  computational and sample constraint, we only consider sets of at
  most two variable.
\item Find the set of variables that gives the maximum average
  lift. Stop if none of the sets has lift or the total number of
  contexts is small.
\item Generate a new tree node for each configuration with lift above
  some threshold and
  generate an ``else'' node for the other configurations. There are
  several ways to pick the threshold. One could use the average lift
  as the threshold or just use a constant as a threshold. Also we
  should take the number of contexts into account. If some
  configuration only has a small number of contexts, we will put it in
  the ``else'' node.
\end{enumerate}
If the problem is optimizable, i.e., the context variables and action
variable are relevant to the reward variable, and the default decision
is not always the best one, then there exists a set of
relevant variables with certain configuration such that some
non-default decision has lift. The important thing to notice is that
lifts are defined for configurations but not the set of
variables themselves. Ideally we should just have a lift table for all
possible configurations. Due to the choice that we only consider
short configurations, we need to look for configurations in a cascading way
to find a long configuration with lift. It is not clear if this choice
is necessary.

This algorithm makes another choice by restricting all children of a
node to have the same set of variable. Ideally we could consider
arbitrary short configurations as children, but this choice might have
some computational benefit. One issue with this choice is not all
configurations for this set of variables have good lift. It could
happen that some configuration corresponding to another set of
variables is able to optimize. The node ``else'' allows the algorithm
to continue exploring the other variables, though it is no longer
possible to consider the configurations involving both current and
other variables.

\begin{remark}
In simplest decision tree, each tree node only represents a single
variable. Here we consider one or two variables as tree node, which is
more general. With the requirement on the number of contexts, we could
consider arbitrary number of variables.
\end{remark}
\begin{remark}
Decision tree could also incorporate the node ``else''. If some
configurations are irrelevant and do not help reduce the entropy, then
the tree should not split there and pool the samples for some later
split.
\end{remark}


\section{Reinforcement Learning}

This is too detailed.


\section{CK12}

\bibliographystyle{IEEEtran}
\bibliography{big}

\end{document}
