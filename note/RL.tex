\documentclass[onecolumn, 12pt]{IEEEtran}
\usepackage{amsfonts}
\usepackage{amsmath}
\usepackage{mathrsfs}
\usepackage{subfig}
\usepackage{algorithm}
\usepackage{algorithmic}
\usepackage{amssymb}
\linespread{1.2}\selectfont
\usepackage{amsthm}
\usepackage{hyperref}
\hypersetup{
  colorlinks=true,
  urlcolor=blue
}

% \newtheorem{thm}{Theorem}[section]
% \newtheorem{lem}{Lemma}[section]
% \newtheorem{prop}{Proposition}[section]
% \newtheorem{corollary}{Corollary}[section]
% \newtheorem{definition}{Definition}[section]

\newcommand\E[1]{\mathbb{E}\left[#1\right]}
\newcommand\EE[2]{\mathbb{E}_{#1}\left[#2\right]}
\newcommand\p[1]{\mathbb{P}\left(#1\right)}
\newcommand\Var[1]{\mathrm{Var}\left[#1\right]}
\newcommand\Cov[1]{\mathbb{Cov}\left[#1\right]}
\newcommand\Tr[1]{\mathrm{Tr}\left[#1\right]}
\newcommand\1[1]{\mathbb{I}_{\left\{#1\right\}}}
\newcommand\sign[1]{\mathrm{sign}(#1)}
\newcommand\cond[3]{#1_{#2|#3}}
\newcommand\N[2]{\mathcal{N}\left(#1, #2\right)}

\newtheorem{lemma}{Lemma}\newtheorem{theorem}{Theorem}\newtheorem{corollary}{Corollary}\newtheorem{remark}{Remark}\newtheorem{conjecture}{Conjecture}
\newtheorem{definition}{Definition}\newtheorem{proposition}{Proposition}
\newtheorem{example}{Example}

\begin{document}

\title{Reinforcement Learning}
\author{Rui Wu}
\maketitle

\section{Monte Carlo method and TD learning}

This section considers several methods for learning the optimal policy. We only consider episode environments.

Let $Q(s, a)$ be the state-action value function and $\pi(a|s)$ be the $\epsilon$-greedy policy derived from $Q(s, a)$. According to the policy improvement theorem, we start from an initial policy and keep improving it towards the optimal policy by iteratively estimating $Q(s, a)$ and updating $\pi(a|s)$. For each episode, we obtain the samples as
\begin{align*}
  s_0, a_0, r_1, s_1, a_1, r_2, s_2, \dots, r_T, s_T (, a_T).
\end{align*}
Then we estimate the sample value for each $(s_t, a_t)$ and update $Q(s, a)$ with these sample values.

\subsection{Batch Monte Carlo}
Monte Carlo method estimates $G_t$ as
\begin{align*}
  G_t = r_{t+1} + \gamma r_{t + 2} + \dots + \gamma^{T-t - 1}r_T. 
\end{align*}
For batch update, we first compute the average reward for each state-action pair as
\begin{align*}
  G(s, a) = \frac{\sum_t{G_t\delta_{s, a}(s_t, a_t)}}{\sum_t{\delta_{s, a}(s_t, a_t)}},
\end{align*}
and then update $Q(s, a)$ as
\begin{align*}
  Q(s, a) \leftarrow Q(s, a) + \alpha(G(s, a) - Q(s, a)). 
\end{align*}

\subsection{Online Monte Carlo}
Online Monte Carlo looks similar to TD learning and is easier to implement. It updates $Q(s, a)$ as
\begin{align*}
  Q(s_t, a_t) \leftarrow Q(s_t, a_t) + \alpha(G_t - Q(s_t, a_t)),
\end{align*}
where $G_t$ is defined as in batch Monte Carlo. Even though this method is called ``online'', it has to wait until an episode ends to actually compute the $G_t$'s.

\subsection{Sarsa}
It is similar to online Monte Carlo except that it computes $G_t$ as
\begin{align*}
  G_t = r_{t+1}+\gamma Q(s_{t+1}, a_{t+1}).
\end{align*}

\subsection{Q-learning}
It is similar to online Monte Carlo except that it computes $G_t$ as
\begin{align*}
  G_t = r_{t+1}+\gamma \max_a Q(s_{t+1}, a).
\end{align*}

\section{On policy and off policy learning}
TODO


\section{Eligibility trace}

This section focuses on appliying eligibility trace to $Q(s, a)$ than $V(s)$ and describes the $\text{Sarsa}(\lambda)$ algorithm in detail.

\subsection{Forward view of $\text{Sarsa}(\lambda)$}

We first generalizes the Sarsa reward $G_t$ to the $n$-step reward
\begin{align*}
  G_t^{(n)} = r_{t + 1} + \gamma r_{t + 2} + \dots + \gamma^{(n - 1)}r_{t +n} + \gamma^nQ(s_{t + n}, a_{t + n}), 
\end{align*}
and define
\begin{align*}
  G_t^\lambda = (1-\lambda)G_t^{(1)} + (1-\lambda)\lambda G_t^{(2)} + \dots + \lambda^{T - t} G_t^{(T - t)}. 
\end{align*}
If we assume an episode has infinite length and uss the assumption that the reward for a termination state remains the same, we can rewrite $G_t^\lambda$ as
\begin{align*}
  G_t^\lambda = (1 - \lambda) \sum_{n = 1}^\infty \lambda^{n - 1}G_t^{(n)}. 
\end{align*}
Then we update $Q(s, a)$ as
\begin{align*}
  Q(s, a) \leftarrow Q(s, a) + \alpha(G_t^\lambda - Q(s, a)). 
\end{align*}

This algorithm is between Monte Carlo and Sarsa. It incorporates a parameter $\lambda$ to balance how much into future we want to consider for reward estimation. However, to implement this forward view algorithm, we have to wait until an episode ends to start updating $Q(s, a)$, which is similar to online Monte Carlo.

\subsection{Backward view of $\text{Sarsa}(\lambda)$}
The backward view addresses the online update issue by introducing eligibility trace. The eligibility trace updates as
\begin{align*}
  Z(s, a) \leftarrow \lambda \gamma Z(s, a) + \delta_{s, a}(s_t, a_t),
\end{align*}
and the algorithm updates $Q(s, a)$ as
\begin{align*}
  Q(s, a) \leftarrow Q(s, a) + \alpha Z(s, a)(R_{t + 1} + \gamma Q(s_{t + 1}, a_{t + 1}) - Q(s_t, a_t)). 
\end{align*}

Note that this algorithm updates all states for each time $t$. This is very expensive compared to Sarsa or Monte Carlo. Is it really worth it?

\subsection{Connection between forward and backward views}
The forward and backward views are equivalent in an episode environments under first visit and batch update, i.e., $Q(s, a)$ is updated in batch after an episode ends. To illustrate this equivalency, we walk through the following example for policy evaluation.
\begin{example}
  Consider an episode as follows
  \begin{align*}
    s_1, a_1, r_2, s_2, a_2, r_3, s_1, a_3, r_4, s_4.
  \end{align*}
  Forward view. The estimated value for the first visit of state $s_1$ is
  \begin{align*}
    G^\lambda(s_1) =& (1 - \lambda) (r_2 + \gamma V(s_2)) \\
    & (1-\lambda) \lambda (r_2 + \gamma r_3  + \gamma^2 V(s_1))\\
    & \lambda^2(r_2 + \gamma r_3 + \gamma^2 r_4 + \gamma^3 V(s_4)).
  \end{align*}
  The value difference is
  \begin{align*}
    \Delta V_F(s_1) = G^\lambda(s_1) - V(s_1).
  \end{align*}
  Backward view. The eligibility trace for $s_1$ updates as
  \begin{align*}
    Z(s_1) = 1 \rightarrow \lambda\gamma \rightarrow 1 + \lambda^2\gamma^2.
  \end{align*}
  The value difference is
  \begin{align*}
    \Delta V_B(s_1) =& 1\cdot (r_2 + \gamma V(s_2) - V(s_1)) \\
    & \lambda\gamma \cdot (r_3 + \gamma V(s_1) - V(s_2))\\
    & (1 + \lambda^2\gamma^2)(r_3 + \gamma V(s_4) - V(s_1)).
  \end{align*}
  It is not difficult to show that $\Delta V_F(s_1) = \Delta V_B(s_1)$. However, this equality is not obvious from the first glance.
\end{example}

\subsection{Other eligibility trace algorithms}
There is no $\text{MC}(\lambda)$ as Monte Carlo method does not do bootstrap. It is possible to do $Q(\lambda)$ but it seems more involved. 


\section{Policy gradient, imitation learning and DAgger}

This section discusses the methods fro learning policies directly without first approximating the value function $Q(s, a)$. A policy parameterized by $\theta$ is denoted as $\pi_\theta(a|s)$. The idea is to create a reward function with the policy and find a good solution for $\theta$ by optimization techniques like stochastic gradient descent.

Note that as the algorithms do not involve $Q(s, a)$, the estimates are all based on the Monte Carlo methods rather than TD learning. In addition, eligibility trace does not apply in these algorithm. 

\subsection{Policy gradient}

Policy gradient uses the expected reward as the reward function,
\begin{align*}
  J(\theta) = \EE{s_0}{V^{\pi_\theta}(s_0)},
\end{align*}
where the expectation is taken over the initial state $s_0$. By expanding $V^{\pi_\theta}(s_0)$ over $a_0$, we get
\begin{align*}
  J(\theta) =& \EE{s_0}{\EE{\pi_{\theta}(a_0|s_0)}{Q^{\pi_\theta}(s_0, a_0)}}\\
  =& \EE{s_0}{\sum_{a_0}{\pi_{\theta}(a_0|s_0)Q^{\pi_\theta}(s_0, a_0)}}.
\end{align*}
The policy gradient theorem states that (not obvious at all as there are two components depending on $\theta$)
\begin{align*}
  \nabla_\theta J(\theta) =& \EE{s_0}{\sum_{a_0}{\nabla_\theta\pi_{\theta}(a_0|s_0)Q^{\pi_\theta}(s_0, a_0)}}\\
  =& \EE{s_0}{\EE{\pi_{\theta}(a_0|s_0)}{Q^{\pi_\theta}(s_0, a_0)\nabla_\theta\log\pi_\theta(a_0|s_0)}}.
\end{align*}
The gradient $\nabla_\theta J(\theta)$ is an expectation over the state-action pair and can be estimated by the Monte Carlo method. In particular, for an episode
\begin{align*}
  s_0, a_0, r_1, s_1, \dots, r_T, s_T(, a_T)
\end{align*}
the REINFORCE algorithm estimates $Q^{\pi_\theta}(s_0, a_0)$ with $G_t$ and updates $\theta$ as
\begin{align*}
  \theta \leftarrow \theta + \alpha G_t\nabla_\theta\log\pi_\theta(a_t|s_t).
\end{align*}

\subsection{Learning from experience}

Supervised learning has been used for many reinforcement learning problems and the primary difference is that reinforcement learning allows the agent to learn from its own experience. In this section we intentionally blur the line between supervised learning and reinforcement learning and consider a general formulation for learning from experience.

The experience is assumed to be provided of the form ${(s_t, a_t, G_t)}_{t = 1}^T$, where $G_t$ quantifies how good the state-action pair $(s_t, a_t)$ is. Let $f$ be the similarity function between two action distributions. Then we can define the reward function for a policy as
\begin{align*}
  J(\theta) = \sum_{t=1}^T{G_tf(\pi_\theta(a|s_t), \delta_{a_t}(a))}, 
\end{align*}
and the corresponding stochastic gradient update for $\theta$ is
\begin{align*}
  \theta \leftarrow \theta + \alpha G_t\nabla_\theta f(\pi_\theta(a|s_t), \delta_{a_t}(a)).
\end{align*}

\subsection{Imitation learning and DAgger}
Imitation learning is one example of learning from experience. The experience ${(s_t^*, a_t^*, 1)}_{t = 1}^T$ is completely generated by an expert policy and all demonstrations by the expert are considered equally good. The similarity function is ususally chose as the negative KL-divergent for discrete actions and negative mean squared error for continuous actions.

DAgger is similar to imitation learning except that the states in the experience ${(s_t, a_t^*, 1)}_{t = 1}^T$ are generated by the policy itself.

\subsection{Policy gradient as learning from experience}

In Andrej Kapathy's blog, policy gradient is described as learning from experience. Consider using the experience of an episode generated by the current policy. Let $G_t$ be the Monte Carlo estimated reward and the similarity function $f$ be the negative KL-divergence, i.e.,
\begin{align*}
  f(\pi_\theta(a|s_t), \delta_{a_t}(a)) = -D(\delta_{a_t}(a)||\pi_\theta(a|s_t)) = \log \pi_\theta(a_t|s_t),
\end{align*}
then the update for learning from experience reduces to policy gradient. However, given the vastly different derivations for policy gradient and learning from experience, this connection is more of a coincidence due to the functional form of the KL-divergence than some inherent relationship.


\section{A3C}
TODO

\end{document}
